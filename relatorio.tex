\documentclass[12pt,a4paper]{article}
\usepackage[utf8]{inputenc}
\usepackage[portuguese]{babel}
\usepackage{graphicx, hyperref, verbatim, multicol}

%\addtolength{\oddsidemargin}{-1.4in}
%\addtolength{\evensidemargin}{-1.4in}
%\addtolength{\textwidth}{2.8in}
\addtolength{\topmargin}{-65 pt}
\addtolength{\textheight}{126 pt}

\title{Relatório 2º Projeto de IA}
\author{Grupo 74 \and Daniel Fernandes 86400 \& Francisco Sousa 86416}
\begin{document}
\maketitle
\begin{multicols}{2}
	\section{Introdução}
	Neste projecto vamos testar alguns algoritmos que lidam com a incerteza no
	mundo. Iremos testar métodos de modelação e inferência com redes Bayesiana,
	bem como métodos de aprendizagem por interação com o mundo.

	\section{Redes Bayesianas}
	%Para efeitos deste projeto as redes serão acíclicas.
	%Para o processo de inferência o algoritmo de eliminação não é necessário
	%mas discussão da diferença entre eles

	%No relatório deve incluir-se:
	%• descrição crítica dos resultados pedidos
	%• descrição dos métodos implementados incluindo vantagens/desvantagens e limitações
	%• discussão da complexidade computacional e possíveis métodos alternativos

	\section{Aprendizagem por Reforço}
	%Para cada um dos ambientes, e por inspecção das trajectórias, fazer uma representação
	%gráfica do ambiente no qual o agente se move (1val).
	%Descrever qual a função de recompensa? (1val)
	%Qual é a politica óptima? (1val)
	%Descrição do forma como o agente se move ( Qual é o impacto de cada acção em cada estado )? (1val)
	%No relatório deve incluir-se também:
	%• descrição crítica dos resultados pedidos
	%• descrição dos métodos implementados incluindo vantagens/desvantagens e limitações
	%• discussão da complexidade computacional e possíveis métodos alternativos

	\end{multicols}
\end{document}